% Options for packages loaded elsewhere
\PassOptionsToPackage{unicode}{hyperref}
\PassOptionsToPackage{hyphens}{url}
%
\documentclass[
]{article}
\usepackage{lmodern}
\usepackage{amssymb,amsmath}
\usepackage{ifxetex,ifluatex}
\ifnum 0\ifxetex 1\fi\ifluatex 1\fi=0 % if pdftex
  \usepackage[T1]{fontenc}
  \usepackage[utf8]{inputenc}
  \usepackage{textcomp} % provide euro and other symbols
\else % if luatex or xetex
  \usepackage{unicode-math}
  \defaultfontfeatures{Scale=MatchLowercase}
  \defaultfontfeatures[\rmfamily]{Ligatures=TeX,Scale=1}
\fi
% Use upquote if available, for straight quotes in verbatim environments
\IfFileExists{upquote.sty}{\usepackage{upquote}}{}
\IfFileExists{microtype.sty}{% use microtype if available
  \usepackage[]{microtype}
  \UseMicrotypeSet[protrusion]{basicmath} % disable protrusion for tt fonts
}{}
\makeatletter
\@ifundefined{KOMAClassName}{% if non-KOMA class
  \IfFileExists{parskip.sty}{%
    \usepackage{parskip}
  }{% else
    \setlength{\parindent}{0pt}
    \setlength{\parskip}{6pt plus 2pt minus 1pt}}
}{% if KOMA class
  \KOMAoptions{parskip=half}}
\makeatother
\usepackage{xcolor}
\IfFileExists{xurl.sty}{\usepackage{xurl}}{} % add URL line breaks if available
\IfFileExists{bookmark.sty}{\usepackage{bookmark}}{\usepackage{hyperref}}
\hypersetup{
  pdftitle={example.R},
  pdfauthor={Torrancew},
  hidelinks,
  pdfcreator={LaTeX via pandoc}}
\urlstyle{same} % disable monospaced font for URLs
\usepackage[margin=1in]{geometry}
\usepackage{color}
\usepackage{fancyvrb}
\newcommand{\VerbBar}{|}
\newcommand{\VERB}{\Verb[commandchars=\\\{\}]}
\DefineVerbatimEnvironment{Highlighting}{Verbatim}{commandchars=\\\{\}}
% Add ',fontsize=\small' for more characters per line
\usepackage{framed}
\definecolor{shadecolor}{RGB}{248,248,248}
\newenvironment{Shaded}{\begin{snugshade}}{\end{snugshade}}
\newcommand{\AlertTok}[1]{\textcolor[rgb]{0.94,0.16,0.16}{#1}}
\newcommand{\AnnotationTok}[1]{\textcolor[rgb]{0.56,0.35,0.01}{\textbf{\textit{#1}}}}
\newcommand{\AttributeTok}[1]{\textcolor[rgb]{0.77,0.63,0.00}{#1}}
\newcommand{\BaseNTok}[1]{\textcolor[rgb]{0.00,0.00,0.81}{#1}}
\newcommand{\BuiltInTok}[1]{#1}
\newcommand{\CharTok}[1]{\textcolor[rgb]{0.31,0.60,0.02}{#1}}
\newcommand{\CommentTok}[1]{\textcolor[rgb]{0.56,0.35,0.01}{\textit{#1}}}
\newcommand{\CommentVarTok}[1]{\textcolor[rgb]{0.56,0.35,0.01}{\textbf{\textit{#1}}}}
\newcommand{\ConstantTok}[1]{\textcolor[rgb]{0.00,0.00,0.00}{#1}}
\newcommand{\ControlFlowTok}[1]{\textcolor[rgb]{0.13,0.29,0.53}{\textbf{#1}}}
\newcommand{\DataTypeTok}[1]{\textcolor[rgb]{0.13,0.29,0.53}{#1}}
\newcommand{\DecValTok}[1]{\textcolor[rgb]{0.00,0.00,0.81}{#1}}
\newcommand{\DocumentationTok}[1]{\textcolor[rgb]{0.56,0.35,0.01}{\textbf{\textit{#1}}}}
\newcommand{\ErrorTok}[1]{\textcolor[rgb]{0.64,0.00,0.00}{\textbf{#1}}}
\newcommand{\ExtensionTok}[1]{#1}
\newcommand{\FloatTok}[1]{\textcolor[rgb]{0.00,0.00,0.81}{#1}}
\newcommand{\FunctionTok}[1]{\textcolor[rgb]{0.00,0.00,0.00}{#1}}
\newcommand{\ImportTok}[1]{#1}
\newcommand{\InformationTok}[1]{\textcolor[rgb]{0.56,0.35,0.01}{\textbf{\textit{#1}}}}
\newcommand{\KeywordTok}[1]{\textcolor[rgb]{0.13,0.29,0.53}{\textbf{#1}}}
\newcommand{\NormalTok}[1]{#1}
\newcommand{\OperatorTok}[1]{\textcolor[rgb]{0.81,0.36,0.00}{\textbf{#1}}}
\newcommand{\OtherTok}[1]{\textcolor[rgb]{0.56,0.35,0.01}{#1}}
\newcommand{\PreprocessorTok}[1]{\textcolor[rgb]{0.56,0.35,0.01}{\textit{#1}}}
\newcommand{\RegionMarkerTok}[1]{#1}
\newcommand{\SpecialCharTok}[1]{\textcolor[rgb]{0.00,0.00,0.00}{#1}}
\newcommand{\SpecialStringTok}[1]{\textcolor[rgb]{0.31,0.60,0.02}{#1}}
\newcommand{\StringTok}[1]{\textcolor[rgb]{0.31,0.60,0.02}{#1}}
\newcommand{\VariableTok}[1]{\textcolor[rgb]{0.00,0.00,0.00}{#1}}
\newcommand{\VerbatimStringTok}[1]{\textcolor[rgb]{0.31,0.60,0.02}{#1}}
\newcommand{\WarningTok}[1]{\textcolor[rgb]{0.56,0.35,0.01}{\textbf{\textit{#1}}}}
\usepackage{graphicx,grffile}
\makeatletter
\def\maxwidth{\ifdim\Gin@nat@width>\linewidth\linewidth\else\Gin@nat@width\fi}
\def\maxheight{\ifdim\Gin@nat@height>\textheight\textheight\else\Gin@nat@height\fi}
\makeatother
% Scale images if necessary, so that they will not overflow the page
% margins by default, and it is still possible to overwrite the defaults
% using explicit options in \includegraphics[width, height, ...]{}
\setkeys{Gin}{width=\maxwidth,height=\maxheight,keepaspectratio}
% Set default figure placement to htbp
\makeatletter
\def\fps@figure{htbp}
\makeatother
\setlength{\emergencystretch}{3em} % prevent overfull lines
\providecommand{\tightlist}{%
  \setlength{\itemsep}{0pt}\setlength{\parskip}{0pt}}
\setcounter{secnumdepth}{-\maxdimen} % remove section numbering
% https://github.com/rstudio/rmarkdown/issues/337
\let\rmarkdownfootnote\footnote%
\def\footnote{\protect\rmarkdownfootnote}

% https://github.com/rstudio/rmarkdown/pull/252
\usepackage{titling}
\setlength{\droptitle}{-2em}

\pretitle{\vspace{\droptitle}\centering\huge}
\posttitle{\par}

\preauthor{\centering\large\emph}
\postauthor{\par}

\predate{\centering\large\emph}
\postdate{\par}

\title{example.R}
\author{Torrancew}
\date{2020-04-26}

\begin{document}
\maketitle

\begin{Shaded}
\begin{Highlighting}[]
\KeywordTok{library}\NormalTok{(budgetly)}


\CommentTok{#The first function that you would want to run is createbudget(). It has two arguments: netincome and savingspercentage}
\CommentTok{#In this example, well create a budget that has a monthly net income of $2000 and a savings percentage of 10%.}
\CommentTok{##reatebudget() creates a data frame}

\NormalTok{data <-}\StringTok{ }\KeywordTok{createbudget}\NormalTok{(}\DecValTok{2000}\NormalTok{,}\DecValTok{10}\NormalTok{)}

\NormalTok{data}
\end{Highlighting}
\end{Shaded}

\begin{verbatim}
##    Month Net.Income Expected.Savings.Percentage Fixed.Expenditures
## 1      1       2000                          10                  0
## 2      2       2000                          10                  0
## 3      3       2000                          10                  0
## 4      4       2000                          10                  0
## 5      5       2000                          10                  0
## 6      6       2000                          10                  0
## 7      7       2000                          10                  0
## 8      8       2000                          10                  0
## 9      9       2000                          10                  0
## 10    10       2000                          10                  0
## 11    11       2000                          10                  0
## 12    12       2000                          10                  0
##    Variable.Expenditures Total.Expenditures Goal.Reached Amount.Saved
## 1                      0                  0            0            0
## 2                      0                  0            0            0
## 3                      0                  0            0            0
## 4                      0                  0            0            0
## 5                      0                  0            0            0
## 6                      0                  0            0            0
## 7                      0                  0            0            0
## 8                      0                  0            0            0
## 9                      0                  0            0            0
## 10                     0                  0            0            0
## 11                     0                  0            0            0
## 12                     0                  0            0            0
##    Amount.Left
## 1            0
## 2            0
## 3            0
## 4            0
## 5            0
## 6            0
## 7            0
## 8            0
## 9            0
## 10           0
## 11           0
## 12           0
\end{verbatim}

\begin{Shaded}
\begin{Highlighting}[]
\CommentTok{##Month - This is a yearly budget organized by the month}

\CommentTok{#Net Income - The amount of money the user expects to make per month}

\CommentTok{#Expected Savings Percentage - The perfect of the user's netcome the wish to save per month}

\CommentTok{#Fixed Expenditures - The amount of money the user spent in fixed expences that month. Use this area for mortage, car payments,}
\CommentTok{#insurance, etc}

\CommentTok{#Variable Expenditures - The amount of money the user spent in variable expences that month. Use this area for subscriptions, emergency purchaces,}
\CommentTok{#one time purchases, gifts for others, etc.}

\CommentTok{#Total Expenditures - The total amount of money the user spent in total that month.}

\CommentTok{#Goal Reached - Tells the user if they reached the savings goal they planned for that month}

\CommentTok{#Amount Saved - The amount of money (if any) the user saved that month}

\CommentTok{#Amount Left - The amount of money the user has after savings}

\CommentTok{#Columns 6-9 are filled out after the user use the function updatebudget()}


\NormalTok{data <-}\StringTok{ }\KeywordTok{updatebudget}\NormalTok{(data, }\DecValTok{1}\NormalTok{, }\DecValTok{500}\NormalTok{, }\DecValTok{400}\NormalTok{)}

\NormalTok{data}
\end{Highlighting}
\end{Shaded}

\begin{verbatim}
##    Month Net.Income Expected.Savings.Percentage Fixed.Expenditures
## 1      1       2000                          10                500
## 2      2       2000                          10                  0
## 3      3       2000                          10                  0
## 4      4       2000                          10                  0
## 5      5       2000                          10                  0
## 6      6       2000                          10                  0
## 7      7       2000                          10                  0
## 8      8       2000                          10                  0
## 9      9       2000                          10                  0
## 10    10       2000                          10                  0
## 11    11       2000                          10                  0
## 12    12       2000                          10                  0
##    Variable.Expenditures Total.Expenditures Goal.Reached Amount.Saved
## 1                    400                900          Yes          900
## 2                      0                  0            0            0
## 3                      0                  0            0            0
## 4                      0                  0            0            0
## 5                      0                  0            0            0
## 6                      0                  0            0            0
## 7                      0                  0            0            0
## 8                      0                  0            0            0
## 9                      0                  0            0            0
## 10                     0                  0            0            0
## 11                     0                  0            0            0
## 12                     0                  0            0            0
##    Amount.Left
## 1         1100
## 2            0
## 3            0
## 4            0
## 5            0
## 6            0
## 7            0
## 8            0
## 9            0
## 10           0
## 11           0
## 12           0
\end{verbatim}

\begin{Shaded}
\begin{Highlighting}[]
\CommentTok{#updatebudget() takes 4 arguments}

\CommentTok{#A data frame (created using the createbudget()), the month to be updated, the fixed expences for that month,}
\CommentTok{#and the variable expenses that month}

\CommentTok{#The function fills out the rest of the data frame}

\CommentTok{#In this example, We will fill in the next two months of the yearly budget}

\NormalTok{data <-}\StringTok{ }\KeywordTok{updatebudget}\NormalTok{(data, }\DecValTok{2}\NormalTok{, }\DecValTok{600}\NormalTok{, }\DecValTok{600}\NormalTok{)}

\NormalTok{data <-}\StringTok{ }\KeywordTok{updatebudget}\NormalTok{(data, }\DecValTok{3}\NormalTok{, }\DecValTok{1300}\NormalTok{, }\DecValTok{500}\NormalTok{)}

\NormalTok{data}
\end{Highlighting}
\end{Shaded}

\begin{verbatim}
##    Month Net.Income Expected.Savings.Percentage Fixed.Expenditures
## 1      1       2000                          10                500
## 2      2       2000                          10                600
## 3      3       2000                          10               1300
## 4      4       2000                          10                  0
## 5      5       2000                          10                  0
## 6      6       2000                          10                  0
## 7      7       2000                          10                  0
## 8      8       2000                          10                  0
## 9      9       2000                          10                  0
## 10    10       2000                          10                  0
## 11    11       2000                          10                  0
## 12    12       2000                          10                  0
##    Variable.Expenditures Total.Expenditures Goal.Reached Amount.Saved
## 1                    400                900          Yes          900
## 2                    600               1200          Yes          600
## 3                    500               1800           No            0
## 4                      0                  0            0            0
## 5                      0                  0            0            0
## 6                      0                  0            0            0
## 7                      0                  0            0            0
## 8                      0                  0            0            0
## 9                      0                  0            0            0
## 10                     0                  0            0            0
## 11                     0                  0            0            0
## 12                     0                  0            0            0
##    Amount.Left
## 1         1100
## 2          800
## 3          200
## 4            0
## 5            0
## 6            0
## 7            0
## 8            0
## 9            0
## 10           0
## 11           0
## 12           0
\end{verbatim}

\begin{Shaded}
\begin{Highlighting}[]
\CommentTok{#To visualize the amount of money the user uses in a month, the can use viewbudget()}
\CommentTok{#viewbudget takes two arguments: a data frame and the month the user wishes to view}

\NormalTok{chart <-}\StringTok{ }\KeywordTok{viewbudget}\NormalTok{(data,}\DecValTok{2}\NormalTok{)}
\end{Highlighting}
\end{Shaded}

\begin{verbatim}
## [1] "Budget Breakdown for the Month 2"
\end{verbatim}

\includegraphics{example_files/figure-latex/unnamed-chunk-1-1.pdf}

\begin{Shaded}
\begin{Highlighting}[]
\NormalTok{chart}
\end{Highlighting}
\end{Shaded}

\begin{verbatim}
## NULL
\end{verbatim}

\begin{Shaded}
\begin{Highlighting}[]
\CommentTok{#In this example, we are looking at the 2nd month of the year}


\CommentTok{#To compare two months to each other use comparebudget()}
\CommentTok{#comparebudget() takes four arugments (three required and one optional): A data frame, the first month, the second month,}
\CommentTok{#and an optional argument - sidebyside}

\KeywordTok{comparebudget}\NormalTok{(data,}\DecValTok{1}\NormalTok{,}\DecValTok{3}\NormalTok{)}
\end{Highlighting}
\end{Shaded}

\begin{verbatim}
## [1] "Your income for month  1 was 2000"
## [1] "Your income for month  3 was 2000"
## [1] "You saved more money in month 1"
\end{verbatim}

\begin{Shaded}
\begin{Highlighting}[]
\CommentTok{#Setting the arugment sidebyside to "YES" creates a data frame with the two months listed}


\NormalTok{months <-}\StringTok{ }\KeywordTok{comparebudget}\NormalTok{(data,}\DecValTok{1}\NormalTok{,}\DecValTok{3}\NormalTok{,}\StringTok{"YES"}\NormalTok{)}

\NormalTok{months}
\end{Highlighting}
\end{Shaded}

\begin{verbatim}
##   Month Net.Income Expected.Savings.Percentage Fixed.Expenditures
## 1     1       2000                          10                500
## 2     2       2000                          10                600
## 3     3       2000                          10               1300
##   Variable.Expenditures Total.Expenditures Goal.Reached Amount.Saved
## 1                   400                900          Yes          900
## 2                   600               1200          Yes          600
## 3                   500               1800           No            0
##   Amount.Left
## 1        1100
## 2         800
## 3         200
\end{verbatim}

\end{document}
